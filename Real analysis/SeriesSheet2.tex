\documentclass{article}
\usepackage{amsmath} % Required for mathematical symbols and fonts
\usepackage{graphicx} % Required for inserting images
\usepackage{tikz}
\usepackage{amssymb} % Required for additional mathematical symbols
\usepackage{amsthm}

\theoremstyle{remark}
\newtheorem*{remark}{Remark}
\usepackage[T1]{fontenc}
\usepackage{lmodern}

\title{Finite and infinite series sheet}
\author{Gallo Tenis, CoGi / to A Mathematical Room}
\begin{document}
\maketitle

\begin{center}
    \textit{Problems on series and products over finite and infinite sets: books of Terence Tao's ``Analysis I'', Donald E. Knuth's ``The Art of Computer Programming''.
    As well as problems of the Putnam Mathematical Competition and the International Math Olympiad.}
\end{center}
\section*{Convergence of series}
\begin{enumerate}
    \item Let \( \sum_{n=m}^{\infty} a_n \) be a formal series of real numbers. Then \( \sum_{n=m}^{\infty} a_n \) converges if and only if, for every real number \( \varepsilon > 0 \), there exists an integer \( N \geq m \) such that
    \[
    \left| \sum_{n=p}^{q} a_n \right| \leq \varepsilon \quad \text{for all } p,q \geq N.
    \]
    $\textbf{Solution.}$
    \begin{enumerate}
        \item[$(\implies)$] If $\sum_{n=m}^{\infty} a_n$ converges then, by definition, the partial sum sequence 
        \[
        (S_N)_{N=m}^{\infty} = \left(\sum_{n=m}^{N}a_n\right)_{N=m}^{\infty}
        \]
        converges to $L$.
        Therefore, for any $\varepsilon > 0$ there exists some $M \geq m$ such that for any $p-1,q \geq M$
        \[
        \vert S_q - S_{p-1} \vert \leq \varepsilon.
        \]
        Without loss of generality assume $p - 1 \leq q$, it follows that
        \[
        \left\lvert \sum_{n=m}^{q}a_n - \sum_{n=m}^{p-1}a_n \right\rvert = \left\lvert \sum_{n=p}^{q}a_n \right\rvert \leq \varepsilon.
        \]
        
        \item[$(\impliedby)$] Doing the reverse process of the direct implication the result follows.
    \end{enumerate}
    \begin{flushright}
        \qed
    \end{flushright}

    \item \textbf{(Zero test)}. Let \( \sum_{n=m}^{\infty} a_n \) be a convergent series of real numbers. Then we must have \( \lim_{n \to \infty} a_n = 0 \). To put this another way, if \( \lim_{n \to \infty} a_n \) is non-zero or divergent, then the series \( \sum_{n=m}^{\infty} a_n \) is divergent.
    \\$\textbf{Solution.}$
    If $\sum_{n=m}^{\infty}a_n$ converges then by last problem, for any $\varepsilon > 0$, there exists some $M \geq m$ such that for any $p,q \geq M$
    \begin{equation}
        \left \lvert \sum_{n=p}^{q} a_n \right\rvert \leq \varepsilon.
    \end{equation}
    Suppose for the sake of contradiction that $\lim_{n\to\infty}a_n = L > 0$ so that for any $\varepsilon > 0$ there exists 
    some $M'$ such that for any $n \geq M'$ 
    \[
    \vert a_n - L \vert < \varepsilon
    \]
    so that in particular $a_n \geq L-\varepsilon$.

    Therefore, for any $p,q \geq K := \max\{M, M'\}$
    \[
    \left \lvert \sum_{n=p}^{q} a_n \right\rvert \geq \left| \sum_{n=p}^{q} L- \varepsilon \right| = (q-p+1)\left| L - \varepsilon \right|.
    \]
    Hence, let $\varepsilon = L/2$, we see that if the above inquality were true 
    \[
    \left \lvert \sum_{n=p}^{q} a_n \right\rvert \geq \frac{(q-p+1)L}{2}.
    \]
    Then let $q = p+1$, it follows that 
    \[
        \left \lvert \sum_{n=p}^{q} a_n \right\rvert \geq L > \varepsilon. 
    \]
    A contradiction of inequality $(1)$.
    \begin{flushright}
        \qed
    \end{flushright}

    \item \textbf{(Absolute convergence test)}. Let \( \sum_{n=m}^{\infty} a_n \) be a formal series of real numbers. If this series is absolutely convergent, then it is also conditionally convergent. Furthermore, in this case we have the triangle inequality
    \[
    \left| \sum_{n=m}^{\infty} a_n \right| \leq \sum_{n=m}^{\infty} |a_n|.
    \]
    $\textbf{Solution.}$
    If $\sum_{n=m}^{\infty}\vert a_n \vert$ converges to $L$, then for any $\varepsilon > 0$ there exists some 
    $M \geq m$ such that for any $p,q \geq M$
    \[
    \left\lvert \sum_{n=p}^{q}\vert a_n \vert \right\rvert \leq \varepsilon.
    \]
    Hence, 
    \[
    \sum_{n=p}^{q}\vert a_n \vert \leq \varepsilon.
    \]
    By triangle inequality over finite series, for any $p,q \geq M$ we have
    \[
    \left\lvert \sum_{n=p}^{q} a_n \right\rvert \leq \sum_{n=p}^{q}\vert a_n \vert.
    \]
    We conclude that $\sum_{n=p}^{q}a_n$ converges. Trivially, we also note that again, by triangle inequality over 
    finite series, each partial sum $S_N = \sum_{n=m}^{N}a_n$ is smaller than or equal to $T_N = \sum_{n=m}^{N}\vert a_n \vert$,
    meaning that by comparison principle $\lim_{N\to\infty}S_N \leq \lim_{N\to\infty}T_N$.
    \begin{flushright}
        \qed
    \end{flushright}

    \item (\textbf{Series laws}).
    \begin{enumerate}
        \item[(a)] If \( \sum_{n=m}^{\infty} a_n \) is a series of real numbers converging to \( x \), and
        \( \sum_{n=m}^{\infty} b_n \) is a series of real numbers converging to \( y \), then
        \( \sum_{n=m}^{\infty} (a_n + b_n) \) is also a convergent series, and converges to \( x + y \).\\
        In particular, we have
        \[
        \sum_{n=m}^{\infty} (a_n + b_n)
        = \sum_{n=m}^{\infty} a_n + \sum_{n=m}^{\infty} b_n.
        \]

        \item[(b)] If \( \sum_{n=m}^{\infty} a_n \) is a series of real numbers converging to \( x \), and
        \( c \) is a real number, then \( \sum_{n=m}^{\infty} (c a_n) \) is also a convergent series, and
        converges to \( cx \).\\
        In particular, we have
        \[
        \sum_{n=m}^{\infty} (c a_n)
        = c \sum_{n=m}^{\infty} a_n.
        \]

        \item[(c)] Let \( \sum_{n=m}^{\infty} a_n \) be a series of real numbers, and let
        \( k \geq 0 \) be an integer. If one of the two series
        \( \sum_{n=m}^{\infty} a_n \) and \( \sum_{n=m+k}^{\infty} a_n \) are convergent,
        then the other one is also, and we have the identity
        \[
            \sum_{n=m}^{\infty} a_n
            = \sum_{n=m}^{m+k-1} a_n + \sum_{n=m+k}^{\infty} a_n.
        \]

        \item[(d)] Let \( \sum_{n=m}^{\infty} a_n \) be a series of real numbers converging to \( x \),
        and let \( k \) be an integer. Then
        \( \sum_{n=m+k}^{\infty} a_{n-k} \) also converges to \( x \).
    \end{enumerate}

    \item \textbf{(Comparison test.)}

    \item \textbf{(Geometric series.)}
    
    \item \textbf{(Cauchy criterion.)}

    \item \textbf{(Root test.)}
    
    \item \textbf{(Ratio test.)}

    \item Let $(a_n)_{n\geq 1}$ be a sequence of positive real numbers such that
    the series $\sum_{n=1}^{\infty}a_n$ converges. Show that the series
    \begin{center}
        $\displaystyle \sum_{n=1}^{\infty}a_n^{\frac{n}{n+1}}$
    \end{center}
    Also converges.\\
    $\textbf{Solution.}$
    Consider the partial sum sequence $(S_N)_{N=m}^{\infty}$ such that $S_n = \sum_{n=m}^{N}a_n^{\frac{n}{n+1}}$.
    We see that 
    \[
    b_n = \frac{n}{n+1} = \frac{(n+1)-1}{n+1} = 1 - \frac{1}{n+1}
    \]
    so that $\lim_{n\to\infty}b_n = 1$.
    By zero test, since $\sum_{n=1}^{\infty}a_n$ converges, necessarily $\lim_{n\to\infty}a_n = 0$.
    We therefore can study the behaviour of the sequence
    \[
    \lim_{n\to\infty} a_n^{b_n}.
    \]
    That results into 
    \[
    \lim_{n\to\infty} \frac{a_n}{a_n^{\frac{1}{n+1}}}.
    \]
    Note that we cannot use limit laws theorem since we do not know whether $a_n^{\frac{1}{n+1}}$ might converge.
    Suppose that it converged to $c \in \mathbb{R}$, then
    \[
    \lim_{n\to\infty}a_n^{b_n} = \frac{1}{c}\lim_{n\to\infty}a_n = 0.
    \]
    Then one thing we shall also verify is that it is bounded above by for example $\frac{1}{n(n+1)}$ thus 
    by comparison principle the requested series converge.



\end{enumerate}
\section*{Series expansions}
\end{document}