\documentclass{article}
\usepackage{amsmath} % Required for mathematical symbols and fonts
\usepackage{graphicx} % Required for inserting images
\usepackage{tikz}
\usepackage{amssymb} % Required for additional mathematical symbols
\usepackage{amsthm}

\theoremstyle{remark}
\newtheorem*{remark}{Remark}
\usepackage[T1]{fontenc}
\usepackage{lmodern}

\title{Convergence Sheet II}
\author{Gallo Tenis / to A Mathematical Room}

\begin{document}

\maketitle

\begin{center}
    \textit{Problems on convergence of sequences, functions and continuity. The problems are from the books of M. Spivak ``Calculus'', T. Tao ``Analysis I'', R. G. Bartle ``Introduction to Real Analysis''
    and Titu Andreescu ``Putnam and Beyond''.}
\end{center}
\section*{Suprema and Infima of sequences}
    \begin{enumerate}
        \item Let \( E \) be a subset of \( \mathbb{R}^* \). Then the following statements are true.

        \begin{enumerate}
            \item[(a)] For every \( x \in E \) we have \( x \leq \sup(E) \) and \( x \geq \inf(E) \).
        
            \item[(b)] Suppose that \( M \in \mathbb{R}^* \) is an upper bound for \( E \), i.e., \( x \leq M \) for all \( x \in E \). Then we have \(\sup(E) \leq M\).
        
            \item[(c)] Suppose that \( M \in \mathbb{R}^* \) is a lower bound for \( E \), i.e., \( x \geq M \) for all \( x \in E \). Then we have \(\inf(E) \geq M\).
        \end{enumerate}
        $\textbf{Solution.}$


        \item \textbf{(Least upper bound property)}. Let
        \( (a_n)_{n=m}^{\infty} \) be a sequence of real numbers, and let
        \( x \) be the extended real number
        \[
        x := \sup (a_n)_{n=m}^{\infty}.
        \]
        Then we have \( a_n \leq x \) for all \( n \geq m \). Also,
        whenever \( M \in \mathbb{R}^* \) is an upper bound for \( a_n \)
        (i.e., \( a_n \leq M \) for all \( n \geq m \)), we have \( x \leq M \).
        
        Finally, for every extended real number \( y \) for which \( y < x \),
        there exists at least one \( n \geq m \) for which
        \[
        y < a_n \leq x.
        \]
        $\textbf{Solution.}$
        The first two propositions follow directly from the definitions of suprema and infima of sequences and sets 
        of extended real numbers.

        For the final proposition, if $x = +\infty$, and $y \in \mathbb{R}$ then given that 
        $a_n \in \mathbb{R}$, $(a_n)$ is divergent. 
        Therefore, for any $K$ there exists some $n \geq m$ such that 
        $a_n \geq K$. In particular there exists some $n_0 \geq m$ such that $a_n \geq K_0 > y$.

        If $x \in \mathbb{R}$ then $y \in \mathbb{R}$ for any $y < x$. 
        Suppose for the sake of contradiction that there existed at least one $y_0 < x$ such that for any $n \geq m$, $a_n \leq y_0$.
        Then immediately we get a contradiction, as $y_0$ is an upper bound smaller than $x$.

        Note that $x,y = \infty$ but $y < x$ would mean that $\infty < \infty$ which is false.
        \begin{flushright}
            \qed
        \end{flushright}

        \item (\textbf{Monotone bounded sequences converge}).
        Let \((a_n)_{n=m}^{\infty}\) be a sequence of real numbers which has some finite upper bound \(M \in \mathbb{R}\), and which is also increasing (i.e., \(a_{n+1} \geq a_n\) for all \(n \geq m\)). Then \((a_n)_{n=m}^{\infty}\) is convergent, and in fact
        \[
        \lim_{n \to \infty} a_n = \sup(a_n)_{n=m}^{\infty} \leq M.
        \]
        $\textbf{Proof.}$
        Since $(a_n)$ is bounded, its supremum cannot be infinite.
        Let $\varepsilon_n$ be the real number $\sup(a_n) - a_n$.
        We see that $(\varepsilon_n)$ is a decreasing positive sequence, has supremum $\sup(a_n)-a_1$.
        It is verified that $\inf(\varepsilon_n) = 0$ since otherwise, if $\inf(\varepsilon_n) = c > 0$ then 
        for any $n \geq 1$, by problem (2)
        \[
        c = \inf(\varepsilon_n) = \inf(\sup(a_n) - a_n) \leq \sup(a_n) - a_n 
        \]
        so that
        \[
        a_n \leq \sup(a_n) - c < \sup(a_n)
        \]
        which contradicts the fact that $\sup(a_n)$ is the least upper bound of $(a_n)$.

        We claim that given any positive integer $k \geq 1$ there exists some $K$ such that for any $n,m \geq K$,
        \[
        \vert a_n - a_m \vert \leq \varepsilon_{k}.
        \]
        To prove this claim we see that for $n,m \geq k$ by symmetry if $n \geq m$, since $(a_n)$ is monotone
        \[
        \vert a_n - a_m \vert = a_n - a_m \leq \sup(a_n) - a_n \leq \sup(a_n) - a_k = \varepsilon_k. 
        \]
        We conclude that suffices letting $K := k$.

        Then, let $\varepsilon > 0$ be any real number, we claim that there exists some $N \geq 1$ such that for any $n \geq N$
        \[
        \vert a_n - \sup(a_n) \vert \leq \varepsilon.
        \]

        To prove this claim we see that since $\inf(\varepsilon_n) = 0$, by problem (2), for any $\varepsilon > 0$ we can find at least one $k$ such that $\varepsilon > \varepsilon_k \geq 0$,
        hence this shows that
        \[
        \vert a_n - a_m \vert \leq \varepsilon
        \]
        for a given $\varepsilon$ and for every $n,m \geq K = k$, so the sequence converges by Cauchy criterion.
        Then, by construction
        \[
        \vert \sup(a_n) - a_n \vert = \varepsilon_n \leq \varepsilon.
        \]
        \begin{flushright}
            \qed
        \end{flushright}

        \item Explain why Proposition 6.3.10 fails when \( x > 1 \). 
        In fact, show that the sequence \((x^n)_{n=1}^\infty\) diverges when \( x > 1 \). 
        (Hint: prove by contradiction and use the identity \((1/x)^n x^n = 1\) and the limit laws in Theorem 6.1.19.) 
        Compare this with the argument in Example 1.2.3; can you now explain the flaws in the reasoning in that example?
        \begin{remark}
            \textbf{Proposition 6.3.10.} \textit{Let } $0 < x < 1$. \textit{Then we have } 
            
            $\lim_{n\to\infty}x^n = 0$.
        \end{remark}
        $\textbf{Solution.}$ Let $x > 1$.
        Suppose for the sake of contradiction that $\lim_{n\to\infty}x^n$ converged to $L \in \mathbb{R}$.
        Consider the sequence $\left(\frac{1}{x}\right)^n$, this sequence converges to $0$ by squeeze theorem with $\frac{1}{n}$.
        Therefore, by limit laws (since both sequences supposedly converge) we should get that 
        \[
        \lim_{n\to\infty}\left(\frac{1}{x}\right)^n\cdot x^n = \lim_{n\to\infty} \left(\frac{1}{x}\right)^n\lim_{n\to\infty}x^n = 0\cdot L = 0.
        \]
        In spite of this, the limit should be equal to 
        \[
        \lim_{n\to\infty}\left(\frac{1}{x^n}\right)\cdot x^n = \lim_{n\to\infty}\frac{x^n}{x^n} = 1.
        \]
        We conclude that $0=1$ which is false. We conclude that $x^n$ does not converge if $x > 1$, note that in this case the test 
        of falseness we used was not the epsilon definition but the limit laws theorem.

        Regarding $Example$ $1.2.3$, that essentially says that for any $x \in \mathbb{R}$
        \[
        L = \lim_{n\to\infty}x^n = \lim_{m+1\to\infty}x^{m+1} = x\lim_{m+1\to\infty}x^{m+1},
        \]
        but by limit laws:
        \[
        x\lim_{m+1\to\infty}x^{m} = xL,
        \]
        concluding that 
        \[
        xL = L.
        \]
        So that $L = 0$ or $x = 1$.
        the two main flaws are: 
        first, assuming that the sequence converges to a finite limit $L$ for any $x$ (that we have shown it does not)
        and the second (related to the first), that we can apply limit laws in possibly divergent sequences. The condition
        to the limit laws theorem to hold is the sequences to converge in first place.
        \begin{flushright}
            \qed
        \end{flushright}

        \item \textbf{(Limits are limit points.)}
        Let \((a_n)_{n=m}^{\infty}\) be a sequence which converges to a real number \(c\). Then \(c\) is a limit point of \((a_n)_{n=m}^{\infty}\), and in fact it is the only limit point of \((a_n)_{n=m}^{\infty}\).
        \\$\textbf{Proof.}$
        \begin{enumerate}
            \item[\textbf{Existence}]
            If $(a_n)$ converges to $c$ then for any $\varepsilon > 0$ there exists an $N \geq m$ such that for any $n \geq N$
            \[
            \vert a_n - c \vert \leq \varepsilon.
            \]
            In order for $c$ to be a limit point, it must happen that for any $\varepsilon > 0$
            and any $N' \geq m$ there exists an $n' \geq N'$ such that
            \[
            \vert a_{n'} - c \vert \leq \varepsilon.
            \]
            Let $n' = \max\{N,N'\}$ (fixed $N$ and variable $N'$) then in particular for any $n \geq n' \geq N$
            \[
            \vert a_n - c \vert \leq \varepsilon \text{ }( (a_n) \text{ converges})
            \]
            and in particular since $n' \geq N'$ and by convergence
            \[
            \vert a_{n'} - c \vert \leq \varepsilon \text{ }( c \text{ is a limit point}).
            \]
            \item[\textbf{Uniqueness}]
            Note that limit are limit points does not guarantee that limit points are limits.
            Let $c,d$ be limit points of $(a_n)$. Then for any $\varepsilon/2 > 0$ and $N \geq m$ there exists $n,n' \geq N$
            such that
            \[
            \vert a_n - c \vert \leq \varepsilon/2 \text{ and } \vert a_{n'} - d\vert \leq \varepsilon/2
            \]
            by symmetry assume $n \geq n'$ then let $N = n'$ so that $n, n' \geq n'$
            \[
            \vert a_n - c \vert + \vert a_{n} - d \vert \leq \varepsilon. 
            \]
            we observe that by triangle inequality
            \[
            \vert c - d \vert \leq \vert c - K\vert + \vert K - d\vert \leq \varepsilon.
            \]
            By the arbitrariness of $\varepsilon$ we conclude that $c = d$.
        \end{enumerate}
        \begin{flushright}
            \qed
        \end{flushright}

        \item Let \( (a_n)_{n=m}^{\infty} \) be a sequence
        of real numbers, let \( L^+ \) be the limit superior of this sequence,
        and let \( L^- \) be the limit inferior of this sequence
        (thus both \( L^+ \) and \( L^- \) are extended real numbers).
        \begin{enumerate}
            \item[(c)] We have 
            \[
                \inf(a_n)_{n=m}^{\infty} \leq L^- \leq L^+ \leq \sup(a_n)_{n=m}^{\infty}.
            \]
        
            \item[(d)] If \( c \) is any limit point of \( (a_n)_{n=m}^{\infty} \),
            then we have 
            \[
                L^- \leq c \leq L^+.
            \]
            
            $L^+$ is an upper bound of $c$ and $L^-$ is a lower bound, so the inequality holds.
        
            \item[(e)] If \( L^+ \) is finite, then it is a limit point of 
            \( (a_n)_{n=m}^{\infty} \). Similarly, if \( L^- \) is finite,
            then it is a limit point of \( (a_n)_{n=m}^{\infty} \).
        
            \item[(f)] Let \( c \) be a real number. If \( (a_n)_{n=m}^{\infty} \)
            converges to \( c \), then we must have \( L^+ = L^- = c \). Conversely,
            if \( L^+ = L^- = c \), then \( (a_n)_{n=m}^{\infty} \) converges to \( c \).
        \end{enumerate}
        $\textbf{Proof.}$
        \begin{enumerate}
            \item 
        \end{enumerate}

        \begin{flushright}
            \qed
        \end{flushright}

        \item \textbf{(Comparison principle.)} Suppose that \( (a_n)_{n=m}^{\infty} \) and 
        \( (b_n)_{n=m}^{\infty} \) are two sequences of real numbers 
        such that \( a_n \leq b_n \) for all \( n \geq m \). 
        Then we have the inequalities:
        \[
            \sup(a_n)_{n=m}^{\infty} \leq \sup(b_n)_{n=m}^{\infty}
        \]
        \[
            \inf(a_n)_{n=m}^{\infty} \leq \inf(b_n)_{n=m}^{\infty}
        \]
        \[
            \limsup_{n \to \infty} a_n \leq \limsup_{n \to \infty} b_n
        \]
        \[
            \liminf_{n \to \infty} a_n \leq \liminf_{n \to \infty} b_n
        \]

        \item \textbf{(Squeeze test.)}
        Let $(a_n)_{n=m}^{\infty}$, $(b_n)_{n=m}^{\infty}$, and $(c_n)_{n=m}^{\infty}$ be sequences of real numbers such that
        \[
        a_n \leq b_n \leq c_n
        \]
        for all $n \geq m$. Suppose also that $(a_n)_{n=m}^{\infty}$ and $(c_n)_{n=m}^{\infty}$ both converge to the same limit $L$. Then $(b_n)_{n=m}^{\infty}$ is also convergent to $L$.

        \item \textbf{(Zero test for sequences.)}
        Let $(a_n)_{n=M}^\infty$ be a sequence of real numbers. Then the limit $\lim_{n \to \infty} a_n$ exists and is equal to zero if and only if the limit $\lim_{n \to \infty} |a_n|$ exists and is equal to zero.
    \end{enumerate}
\section*{Subsets of the real line}
\section*{Limits and continuity}
\end{document}