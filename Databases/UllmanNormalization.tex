\documentclass{article}
\usepackage{amsmath} % Required for mathematical symbols and fonts
\usepackage{graphicx} % Required for inserting images
\usepackage{tikz}
\usepackage{tikz-cd}
\usepackage{amssymb} % Required for additional mathematical symbols
\usepackage{amsthm}

\newtheorem{lemma}{Lemma}
\usepackage{mathtools}

\theoremstyle{remark}
\newtheorem*{remark}{Remark}
\usepackage[T1]{fontenc}
\usepackage{lmodern}

\title{Normalization Theory sheet I}
\author{Gallo Tenis}
\begin{document}

\maketitle

\begin{center}
    \textit{Exercises in normalization theory from J. Ullman's ``Database Systems - The Complete Book''.
    This week's topics are closures of functional dependencies sets, projections of FD's sets, minimal bases, BCNF and 3NF. 
    }
\end{center}
\section*{Section 3.2.}
\begin{enumerate}
    \item \textbf{3.2.5} Show that if a relation has no attribute that is functionally determined by all the other attributes, then the relation has no nontrivial FD's at all.\\
    $\textbf{Solution.}$ Let $R$ be a relation schema. By hypothesis, for any set of attributes $X \in R$, $X^+ = X$.
    So if a nontrivial FD such as $X \longrightarrow Y$ existed, it would imply that $Y \nsubseteq X$, so $X^+ = XY$, a contradiction.
    \begin{flushright}
        \qed
    \end{flushright}

    \item \textbf{3.2.6} Let \( X \) and \( Y \) be sets of attributes. Show that if \( X \subseteq Y \), then \( X^+ \subseteq Y^+ \), where the closures are taken with respect to the same set of FD's.\\
    $\textbf{Solution.}$ Since $X \subseteq Y$ and $Y \subseteq Y^+$, then necessarily $X \subseteq Y^+$. Since $X \subseteq X^+$, we see that each element of $X^+ \cap X$ is in $Y^+$.
    With this in mind, we have shown the inclusion in the trivial DF's case.

    Now for the case of nontrivial dependencies $x \in X^+ \setminus X$ we have to show that $x \in Y^+$.
    Since $x$ is nontrivial then there exists some $z \in X$ such that $x \nsubseteq z$ and
    \begin{center}
        $z \longrightarrow x$.
    \end{center}
    Since $z \in Y$ as well, we conclude that $x \in Y^+$.
    \begin{flushright}
        \qed
    \end{flushright}

    \item \textbf{3.2.7} Prove that \( (X^+)^+ = X^+ \).\\
    $\textbf{Solution.}$
    Since it is obvious that $X^+ \subseteq (X^+)^+$ (the closure of a FD set always contains itself), remains
    showing that $(X^+)^+ \subseteq X^+$.

    Suppose that there existed a FD $x \in (X^+)^+$ such that $x \notin X^+$.
    Then there exists an $y \in X^+$ such that 
    \begin{center}
        $y \longrightarrow x$.
    \end{center}
    So there must also exist a $z \in X$ such that 
    \begin{center}
        $z \longrightarrow y$,
    \end{center}
    hence, by transitivity $z \longrightarrow x$. We conclude that $x \in X^+$.
    \begin{flushright}
        \qed
    \end{flushright}

    \item \textbf{3.2.8} We say a set of attributes \( X \) is \textit{closed} (with respect to a given set of FD’s) if \( X^+ = X \). Consider a relation with schema \( R(A, B, C, D) \) and an unknown set of FD’s. If we are told which sets of attributes are closed, we can discover the FD’s. What are the FD’s if:

    \begin{itemize}
        \item[(a)] All sets of the four attributes are closed.
        \item[(b)] The only closed sets are $\varnothing$ and $\{A,B,C,D\}$.
        \item[(c)] The closed sets are $\varnothing$, $\{A,B\}$, and $\{A,B,C,D\}$. 
    \end{itemize}
    $\textbf{Solution.}$
    \item[(a)] By exercise 3.2.5 $R$ has no non-trivial dependencies. So $\{A,B,C,D\}^+ = \{A,B,C,D\}$.
    \item[(b)] We claim that there does exist at least one superkey of the relation.
    First we note that the unordered pairs of attributes is \[ \binom{4}{2} > 4,\] so there must exist at least two pairs $X$, $Y$ such that $X \cap Y \neq \varnothing$ (by pigeonhole principle).
    In other words, since the closure of any attribute is at least two (they are not closed) we see that  
    \begin{center}
        $X^+ = XZ$ and
        $Y^+ = YZ$
    \end{center}
    or
    \begin{center}
        $X^+ = XZ$ and
        $Z^+ = ZY$
    \end{center}
    So in the first case
    \begin{center}
        $X \longrightarrow Z$ and $Y \longrightarrow Z$ 
    \end{center}
    then
    \begin{center}
        $XY \longrightarrow Z$.
    \end{center}
    In fact, since neither $Z$ is closed, $Z \longrightarrow T$, hence by transitivity $XY \longrightarrow T$, we conclude 
    that $(XY)^+ = XYZT = ABCD$.

    In the second case, we get that $X$ is a superkey by transitivity.

    So the FD's will look like $\{AB \longrightarrow C, C\longrightarrow D\}$ or $\{A \longrightarrow B, B \longrightarrow C, C\longrightarrow D, D\longrightarrow A\}$ with 
    respect to both cases.

    \item[(c)] Consider the sub relations $R_1 = \pi_{A,B}(R)$ and $R_2 = \pi_{C,D}(R)$.
    For $R_1$, since $A^+ \neq A$ and $B^+ \neq B$ but $(AB)^+ = AB$ it will suffice letting $A \longrightarrow B$ and $B \longrightarrow A$.
    In fact, this is the only way it can be done; as if we let $A \longrightarrow X$ with $X \subseteq R_2$ then
    \begin{center}
        $AB \longrightarrow A \longrightarrow X$, so $(AB)^+ = ABX \neq AB$.
    \end{center}
    
    For $R_2$ we have that $C^+ \neq C$ and $D^+ \neq D$ and that $(CD)^+ \neq CD$, since these are not closed.
    We then see that $CD$ is a superkey by seeing that since
    \begin{center}
        $CD \longrightarrow C$, $CD \longrightarrow D$, $C \longrightarrow X$, $D \longrightarrow Y$ (with $X$ and $Y$ not necessarily different),
    \end{center}
    then 
    \begin{center}
        $CD \longrightarrow XY$.
    \end{center}
    And if $XY = X$, then since $(CD)^+ = CDX$, but since $CDX$ is not closed, by transitivity $(CD)^+ = CDXZ = ABCD$.
    Then, the FD's for $R_2$ will be of the form $\{C \longrightarrow D, D\longrightarrow A\}$, $\{C \longrightarrow A, D \longrightarrow B\}$ 
    or $\{C \longrightarrow A, D\longrightarrow A, CD \longrightarrow B\}$.
    \begin{flushright}
        \qed
    \end{flushright}

    \item \textbf{Exercise 3.2.11} Show that if an FD \( F \) follows from some given FD’s, then we can prove \( F \) 
    from the given FD's using Armstrong's axioms (defined in the box ``A Complete Set of Inference Rules'' in Section 3.2.7). \textit{Hint:} Examine Algorithm 3.7 and show how each step of that algorithm can be mimicked by inferring some FD’s by Armstrong’s axioms.\\
    $\textbf{Solution.}$
    
\end{enumerate}
\end{document}